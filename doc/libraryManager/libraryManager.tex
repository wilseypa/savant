
\documentclass[11pt]{report}

\usepackage{alltt,color,fullpage,psfig}

\newcommand{\version}{0.9}

\begin{document}

\title{
\textbf{SAVANT Library Managment} \\
(Documentation for version \version)}

\author{
\emph{Radharamanan Radhakrishnan} and \emph{Philip A.  Wilsey} \\
Computer Architecture Design Laboratory \\
Dept of ECECS, PO Box 210030 \\
Cincinnati, OH  45221--0030 \\
\texttt{\{ramanan,paw\}@ececs.uc.edu}
}

\date{}

\maketitle

\vspace*{6in}

\noindent
Copyright $\copyright$ 1998 The University of Cincinnati.  All
rights reserved.  

\bigskip

\noindent
Published by the University of Cincinnati \\
Dept of ECECS, PO Box 210030 \\
Cincinnati, OH  45221--0030 USA 

\bigskip

\noindent
Permission is granted to make and distribute verbatim copies of
this manual provided the copyright notice and this permission notice
are preserved on all copies.

\medskip
\noindent
Permission is granted to copy and distribute modified versions of this
manual under the conditions for verbatim copying, provided that the entire
resulting derived work is distributed under the terms of a permission
notice identical to this one.

\medskip
\noindent
Permission is granted to copy and distribute translations of this manual
into another language, under the above conditions for modified versions.

\newpage

\tableofcontents

\chapter{Introduction}

The ability of separating a VHDL description into separate independent
libraries necessitate the existence of a library manager.  As specified by
the VHDL language reference manual (LRM) (chapter 11), parts of a VHDL
description can be separately compiled into a library and accessed in
other parts of the VHDL description.  The library manager has been
designed to take care of this functionality.  It allows the user to create
design libraries and helps the parser locate design libraries that are
referenced in other VHDL descriptions.

The following sections describe the specific aspects of the language and
describes the rules that the library manager must conform to.  The next
chapter deals with the specifics of the implementation and describes the
methods used to invoke the library manager.

\section{Design Units}

Certain constructs in a VHDL source file may be independently analyzed and
inserted into a design library; these constructs are called \texttt{design
units}.  One or more design units in a sequence comprise a \texttt{design
file}.  Design units in a design file are analyzed in the textual order of
their appearance in the design file.  Analysis of a design unit defines
the corresponding library unit in a design library.  A \texttt{library
unit} is either a primary unit or a secondary unit.  A secondary unit is a
separately analyzed body of a primary unit resulting from a previous
analysis.  The BNF notation of the constructs are given below :
 
\begin{alltt}
design\_file ::= design\_unit { design\_unit }

design\_unit ::= context\_clause library\_unit

library\_unit ::= primary unit | secondary unit

primary\_unit ::= entity\_declaration | configuration\_declaration 
                                    | package\_declaration 

secondary\_unit ::= architecture\_body | package\_body
\end{alltt}

\section{Design Libraries}

A \texttt{design library} is an implementation-dependent storage facility
for previously analyzed design units.  A given implementation is required
to support any number of design libraries.

\begin{alltt}
library\_clause ::= LIBRARY logical\_name\_list;

logical\_name\_list ::= logical\_name {, logical\_name}

logical\_name ::= identifier

context\_clause ::= { context\_item }

content\_item ::= library\_clause | use\_clause
\end{alltt}

A library clause defines logical names for design libraries in the host
environment.  A library clause appears as part of a context clause at the
beginning of a design unit.  There is a certain region of text called the
\texttt{scope} of a library clause;this region starts immediately after
the library clause, and it extends to the end of the declarative region
associated with the design unit in which the library clause appears.
Within this scope each logical name defined by the library clause is
directly visible, except where hidden in an inner declarative region by a
homo-graph of the logical name according to the rules in Section 10.3 of
the LRM.

There are two classes of design libraries: working libraries and resource
libraries.  A \texttt{working library} is the library into which the
library unit resulting from the analysis of design unit is placed.  A
\texttt{resource library} is a library containing library units that are
referenced within the design unit being analyzed.  Only one library may be
the working library during the analysis of any given design unit; in
contrast, any number of libraries (including the working library itself)
may be resource libraries during such an analysis.

For a given library logical name, the actual name of the corresponding
design libraries in the host environment may or may not be the same.  A
given implementation must provide some mechanism to associate a library
logical name with a host-dependent library.  Such a mechanism is not
defined by the language.  The current implementation of the library
manager supports such a scheme for the working library alone.  All
reference libraries's logical names are assumed to have the same logical
and actual name in the host environment.

Every design unit except package STANDARD is assumed to contain the
following implicit context items as part of its context clause:

\begin{alltt}
        library STD, WORK; use STD.STANDARD.all;
\end{alltt}

Library logical name STD denotes the design library in which the package
STANDARD and package TEXTIO reside;these are the only standard packages
defined by the language.  The use clause makes all declarations within
package STANDARD directly visible within the corresponding design unit.
Library logical name WORK denotes the current working library during a
given analysis.

\section{Order of Analysis}

The rules defining the order in which design units can be analyzed are
direct consequences of the visibility rules.  In particular:

\begin{enumerate}

\item
A primary unit whose name is referenced within a given design unit must be
analyzed prior to the analysis of the given design unit.

\item
A primary unit must be analyzed prior to the analysis of any corresponding
secondary unit.

\end{enumerate}

In each case, the second unit \texttt{depends on} the first unit.  The
order in which design units are analyzed must be consistent with the
partial ordering defined by the above rules.

A given library unit is potentially affected by a change in any library
unit whose name is referenced within the given library unit.  A secondary
unit is potential affected by a change in its corresponding primary unit.
A secondary unit is potentially affected by a change in its corresponding
primary unit.  If a library unit is changed (e.g., by reanalysis of the
corresponding design unit), then all library units that are potentially
affected by such a change become obsolete and must be reanalyzed before
they can be used again.

\chapter{User Manual}

This chapter discusses the methods used to invoke the library manager.  It
also describes some of the naming issues used by the library manager.

\section{Invoking the Library Manager}

There is only one command line parameter to the \texttt{scram} executable
that explicitly deals with the library manager.

\begin{enumerate}

\item
The \texttt{-design-library-name} option is used when the user wants to
create a design library.  Along with the "-design-library-name" flag, the
user must provide the actual design library name.  Please note that this
actual design library name must be used when the design library is used as
a reference library in another design unit.  When \texttt{scram} is
invoked with this option, for each design unit it parses, it downloads the
VHDL for this design unit into the specified directory.  Currently the
parser writes VHDL~\footnote{Code-generation also writes C++ code into
this directory.} to this directory.  In the next release, the file
intermediate representation(FIR) will be written out to the disk instead
of VHDL source.

\end{enumerate}

In addition, to writing out the VHDL source for each design unit (primary
units are placed within the specified directory and the secondary units
are placed one directory level lower than their corresponding primary
units), the library manager writes out a Makefile for the design library
directory.  C++ code generation is also done, so that the design library
may be archived into a library file (lib.a).

The name mangling scheme used by the library manager is:

\begin{alltt}

<design\_library\_name> becomes <design\_library\_name>.\_savant\_lib/

<primary\_unit\_name> becomes   <primary\_unit\_name>\_entity.vhdl 
                            | <primary\_unit\_name>\_configuration.vhdl 
                            | <primary\_unit\_name>\_package.vhdl

Secondary units are placed into a separate directory within the design
library directory.  The naming convention for this separate directory is
as follows :

Directory name is <primary\_unit\_name>\_secondary\_units.\_savant\_lib/

<secondary\_unit\_name> becomes 
          <primary\_unit\_name>\_<secondary\_unit\_name>\_arch.vhdl
        | <primary\_unit\_name>\_<secondary\_unit\_name>\_package\_body.vhdl
\end{alltt}

The name mangling scheme is used to differentiate between names that are
often used repetitively in several design units.  These names may conflict
with each other, when all the design units are being archived into a
single design library.  The same mangling scheme is used by the parser
(with the help of the library manager) when it is looking for design units
in a particular design library.

An example of a typical invocation of the library manager is:

\begin{alltt}
        scram -design-library-name "SHARC" SHARC\_datatype.vhdl
        or
        scram -no-library-manager -publish-vhdl SHARC\_datatype.vhdl
\end{alltt}

As mentioned earlier, every implementation has a working library
associated with every library unit that is analyzed.  \texttt{scram}
auto-magically creates a "work" directory whenever it is invoked.
Although the current version of the library manager archives the VHDL
source, the next version of the library manager will attempt to archive
analyzed design units in the file intermediate representation (FIR).  The
FIR is designed specifically for this reason.  As the FIR is an
intermediate form of the abstract parse tree, it is more efficient to
store and reread FIR than it is to store and reparse VHDL.

\chapter*{Acknowledgments}

This research has been conducted with the participation of many
investigators.  While not an complete list, the following individuals
have made notable direct and/or indirect contributions to this effort
(in alphabetical order):
Jeff Carter,
Harold Carter,
Praveen Chawla,
John Hines,
Herb Hirsch,
Timothy J.  McBrayer,
Greg Peterson,
Al Scarpelli,
Mike Shellhause and
John Willis.
We wish to express our sincerest gratitude for the time that you spent
reviewing and commenting on our designs.

This research was supported in part by MTL Systems, Inc and the Wright
Laboratory at Wright-Patterson AFB.  Thank you for your support.

\end{document}
