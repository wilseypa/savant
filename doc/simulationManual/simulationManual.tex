\documentclass[12pt]{report}

\usepackage{alltt,color,fullpage,psfig-dvips}

\newcommand{\version}{1.0}

\begin{document}

\title{
\textbf{Using SAVANT for VHDL Simulation} \\
(Documentation for version \version)}

\author{
\emph{Radharamanan Radhakrishnan} and \emph{Philip A.  Wilsey} \\
Computer Architecture Design Laboratory \\
Dept of ECECS, PO Box 210030 \\
Cincinnati, OH  45221--0030 \\
\texttt{savant@ececs.uc.edu}
}

\date{}

\maketitle

\vspace*{6in}

\noindent
Copyright $\copyright$ 1995--1999 The University of Cincinnati.  All
rights reserved.  

\bigskip

\noindent
Published by the University of Cincinnati \\
Dept of ECECS, PO Box 210030 \\
Cincinnati, OH  45221--0030 USA 

%\bigskip

\noindent
Permission is granted to make and distribute verbatim copies of
this manual provided the copyright notice and this permission notice
are preserved on all copies.

\medskip
\noindent
Permission is granted to copy and distribute modified versions of this
manual under the conditions for verbatim copying, provided that the entire
resulting derived work is distributed under the terms of a permission
notice identical to this one.

\medskip
\noindent
Permission is granted to copy and distribute translations of this manual
into another language, under the above conditions for modified versions.

\newpage

%\tableofcontents
\chapter{Introduction}

So you have successfully installed SAVANT, TyVIS and WARPED software
distributions and are now ready to use the software for VHDL
simulation. If you have not done this, please follow the installation
directions in the \textsf{README}, \textsf{INSTALL}, and
\textsf{INSTALL-FOR-SIMULATION} text files available in \texttt{savant}
directory\footnote{Also available at
\texttt{http://www.ececs.uc.edu/$\sim$paw/savant}}. Please follow these 
directions and install the software before proceeding further.

This document attempts to give the reader step-by-step guidelines on
how to perform (sequential or parallel) simulation of VHDL examples. The
easiest method to do this is to use a VHDL example and walk the user
through the steps.  Atleast some VHDL knowledge is required to understand
the following description.  So without further ado, lets study the sample
VHDL example (\texttt{test.vhdl}) that we will use to describe the
simulation process:

\section{Example Program - \texttt{test.vhdl}}

\begin{verbatim}
-- *******************************************************************
-- Sample VHDL Simulation Demonstration Program - test.vhdl
-- *******************************************************************

1 library ieee;
2 use ieee.std_logic_1164.all;

3 library std;
4 use std.textio.all;

5 entity myEntity is
6  -- empty entity
7 end myEntity;

8 architecture behavioral of myEntity is
  
9   begin
    
10    myProcess: process is
  
11     variable inputOne : std_logic_vector(0 to 1) := ('0','0'); 
12     variable inputTwo : std_logic_vector(0 to 1) := ('1','1'); 

13     variable traceLine : LINE;
14     variable space : string(1 to 2) := "  ";

15     file RESULT_FILE: text open WRITE_MODE is "results.out";

16     begin

17       write(traceLine, To_bitvector((inputOne and inputTwo)));
18       write(traceLine, space);
19       write(traceLine, To_bitvector((inputOne or inputTwo)));
20       write(traceLine, space);
21       write(traceLine, To_bitvector((inputOne xor inputTwo)));
22       writeLine(RESULT_FILE, traceLine);

23       wait;

24     end process myProcess;

25 end architecture behavioral;  

\end{verbatim}

As can be seen from the VHDL source code, the example instantiates a
process which performs \texttt{and}, \texttt{or} and \texttt{xor}
operations (Lines 17, 19, and 21) on two standard logic vectors
(\texttt{inputOne} and \texttt{inputTwo}). In addition, it writes out the
result of the operations (after converting the standard logic vectors to
bit vectors by calling the \textsf{To\_bitvector} function that is part of
the \texttt{std\_logic\_1164} library) to a file (\texttt{results.out}).

\section{Guidelines for simulating VHDL Programs}

The installation process will build \texttt{scram} (the SAVANT VHDL
analyzer) in a user specified directory (usually in
\texttt{savant/bin}). The analyzer has the following command line
parameters that the user needs to be familiar with: 

\begin{verbatim}
mymachine.home.edu <user_specified_dir>/savant/bin> scram
savant 1.0 (March 8, 1999)
Copyright(C) 1993-1999 The University of Cincinnati.
All rights reserved.

Usage <user_specified_dir>/savant/bin/scram -options <files>
Where options is one of the following:
  -debug-symbol-table    print out debugging info relating to symbol table
  -design-library-name   design library name
  -publish-vhdl          publish VHDL
  -publish-cc            publish c++
  -no-file-output        send publish_cc output to stdout instead of files
  -warranty-info         print information about (lack of) warranty
  -static-elaborate      statically elaborate the specified design
  -partitions            specify # of partitions for elaborated design
  -entity                specify entity to use
  -arch                  specify architecture to use
  -config                specify configuration to use (if needed)
  -alg                   specify static partitioning algorithm to use
  -v                     verbose output
  -help                  print this message

\end{verbatim}

Assuming that the installation process completed successfully, the
following steps detail the process by which an user can use the SAVANT
software for simulating VHDL examples.

\begin{enumerate}

\item Alias \texttt{<user\_specified\_dir>/savant/bin/scram} to
      \texttt{scram}. This is just for convenience so that the user can
      access the \texttt{scram} executable from any directory.

\item The example VHDL file (\texttt{test.vhdl}) imports two libraries
      (Line 1 and 3: \texttt{ieee} and \texttt{std}) and uses these
      libraries just like include files in C/C++. However, the libraries
      need to be handled differently from normal include files. The
      \texttt{std} library gets automatically included by the
      \texttt{scram} parser and the user need not do anything about this
      library. However, the \texttt{ieee} library needs to be
      analyzed/parsed before the example VHDL file can be analyzed. To do
      this, perform the following steps:

      \begin{itemize}
	
	\item [Step 1:] Locate the \texttt{std\_logic\_1164.vhdl} source
	VHDL file. This file needs to analyzed before \texttt{test.vhdl}
	can be analyzed.  This file isn't distributed with savant as it is
	copyighted by IEEE and not freely redistributable.  Note too, that
	versions of this file (and other IEEE libraries) shipped with
	commercial analyzers may vary in subtle or not so subtle ways from
	IEEE's definition standard library.  For instance, the
	std\_logic\_1164 from Synopsys requires other libraries that define
	things that aren't part of the standard.

        \item [Step 2:] Look in the \texttt{std\_logic\_1164.vhdl} source
        VHDL file to see if it in turn includes any other libraries. If it
        does, then recursively execute this step. 

        If you look at the \texttt{std\_logic\_1164.vhdl} source, you might
        see that it includes a package (\texttt{attributes}) from another
        library called \texttt{synopsys}. So we will need to locate the
        \texttt{attributes.vhdl} file which contains the ``attributes''
        package used by the \texttt{std\_logic\_1164.vhdl} library.

	\item [Step 3:] Having resolved the include dependencies of the
	libraries to the \texttt{attributes.vhdl} file, we are now ready
	to create the \texttt{synopsys} and \texttt{ieee}
	design\footnote{For more information about design libraries and
	how they are handled in SAVANT, please refer to the documentation
	on the library manager (available at
	\texttt{http://www.ececs.uc.edu/$\sim$paw/savant})} libraries.

        Analyze the \texttt{attributes.vhdl} file into the
        \texttt{synopsys} design library by executing the following
        command:\\
       
	\texttt{scram -publish-cc -design-library-name synopsys
        attributes.vhdl}\\

	This will result in the creation of a design library named
	\texttt{synopsys.\_savant\_lib} and the code generation of TyVIS
	compliant C++ source files in the current working directory. If
	you look inside this directory, it will contain code-generated
	VHDL files, C++ files and a Makefile. This directory is a
	self-contained unit and the user need not do anything further with
	this directory.

	\item [Step 4:] Repeat Step 3 for the \texttt{ieee} design
	library. Since the \texttt{synopsys} library is now available, the
	\texttt{std\_logic\_1164.vhdl} source file can now be analyzed
	into the \texttt{ieee} design library. Hence, executing the
	following command:\\
     
	\texttt{scram -publish-cc -design-library-name ieee
        std\_logic\_1164.vhdl}\\

	will result in the creation of a design library named
	\texttt{ieee} and the code generation of TyVIS compliant C++
	source files in the newly created design library directory. If you
	look inside this directory, it will contain code-generated VHDL
	files, C++ files and a Makefile. Again, this directory is a
	self-contained unit and the user need not do anything further with
	this directory.

      \end{itemize}

\item We are now ready to analyze the example VHDL program,
      \texttt{test.vhdl} given that both \texttt{ieee} and the
      \texttt{synopsys} design libraries are now available. Execute the
      following command:\\

      \texttt{scram -publish-cc test.vhdl}

\item This will result in the creation of the \texttt{work.\_savant\_lib}
      directory. This is a work directory automatically created by
      \texttt{scram} and contains code generated VHDL and C++ files. It
      also contains the top level Makefile that is cognizant of the other
      two design libraries (\texttt{ieee} and \texttt{synopsys}). Now all
      the user needs to do, is to execute the following command:\\

      \texttt{cd work.\_savant\_lib};\\
      \texttt{make depend; make}

\item The execution of the aforementioned commands will result in the
      compilation of all the code-generated C++ files (including those in
      the \texttt{ieee} and \texttt{synopsys} design libraries) and the
      creation of the executable
      (\texttt{work\_Dmyentity\_work\_Dmyentity\_Dbehavioral}). This will
      be the final simulation executable that the user needs to execute to
      run the simulation. Executing the following command:\\

      \texttt{work\_Dmyentity\_work\_Dmyentity\_Dbehavioral} \\

      will result in the initiation of a sequential simulation (assuming
      the SAVANT software was configured for sequential execution; the
      SAVANT software can be configured for parallel execution
      also\footnote{Please refer to the \textsf{INSTALL-FOR-SIMULATION}
      document for instructions on how to configure the SAVANT software
      for parallel execution.}) of the VHDL source file. Since the
      original VHDL file writes the output of the logical operations to a
      file (\texttt{results.out}), the last step is to ascertain if the
      simulation executed correctly and the correct results were written
      to the file.

\item Looking at the \texttt{results.out} file, we see the following
      result:\\

	 \texttt{00  11  11}\\

      So, is this result right? (\texttt{00 and 11}) results in
      \texttt{00}. (\texttt{00 or 11}) results in \texttt{11}.
      (\texttt{00 xor 11}) results in \texttt{11}. So, the output is
      correct implying that the simulation executed correctly and that
      concludes our little simulation demonstration.

\end{enumerate}

\section*{Acknowledgments}

This research has been conducted with the participation of many
investigators.  While not an complete list, the following individuals have
made notable direct and/or indirect contributions to this effort (in
alphabetical order):

Jeff Carter, Harold Carter, Praveen Chawla, Malolan Chetlur, John Hines,
Herb Hirsch, Dale E. Martin, Timothy J. McBrayer, Greg Peterson, Phani
Putrevu, Radharamanan Radhakrishnan, Umesh Kumar V. Rajasekaran, Dhananjai
Madhava Rao, Al Scarpelli, Mike Shellhause, Krishnan Subramani,
Swaminathan Subramanian, Narayanan V. Thondugulam, and John Willis.

We wish to express our sincerest gratitude for the time that you spent
reviewing and commenting on our designs.  This research was supported in
part by the Wright Laboratory at Wright-Patterson AFB, DARPA, and MTL
Systems, Inc.  Thank you for your support.

\end{document}
